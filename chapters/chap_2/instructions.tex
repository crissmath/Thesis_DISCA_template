\chapter{Instructions}

\section{Writing and Compiling}
\subsection{Document Metadata}
Fill in your title, author name, advisors, and publish date in the preamble of the main \texttt{.tex} file.

\subsection{Content}

\begin{itemize}
    \item Write or paste your chapters' content into individual \texttt{.tex} files within the \texttt{chapters} folder.
    \item For acknowledgements and abstract, fill in \texttt{chapters/1\_acknowledgements.tex} and \texttt{chapters/2\_abstract.tex} respectively.
\end{itemize}

\subsection{Compiling}
\begin{itemize}
    \item Overleaf compiles your project automatically. You can see the PDF preview on the right-hand side of the screen.
    \item If there are any compilation errors, Overleaf will notify you. Click on the 'Logs and output files' button (next to 'Recompile') to view detailed error messages.
\end{itemize}

\subsection{Images and Graphics}
\begin{itemize}
    \item Upload your image files to the \texttt{logos} folder.
    \item Refer to these images in your document using \textbackslash includegraphics\{filename\}.
\end{itemize}

\subsection{Bibliography}
\begin{itemize}
    \item Update your bibliography file (\texttt{biblio.bib}) with your citation entries.
    \item Overleaf supports BibTeX out of the box, so your bibliography should update automatically in the document preview.
\end{itemize}

\subsection{Acronyms}
List your acronyms in the \texttt{chapters/3\_acronym.tex} file, which will be included in the main document.

\subsection{Finalizing}

\subsubsection{Proofreading and Revisions}
\begin{itemize}
    \item Carefully proofread your thesis multiple times. Use Overleaf's rich text mode for an easier editing experience if preferred.
    \item Make revisions directly in Overleaf. The preview will update automatically.
\end{itemize}

\subsubsection{Download PDF}
Once you're satisfied with the document, download the final PDF by clicking the 'Download PDF' button in the top panel.

\subsubsection{Collaboration}
If you wish to collaborate with others (like your advisors or peers), use Overleaf's 'Share' feature to invite them to view or edit the project.

\subsubsection{Version Control}
Overleaf offers version control capabilities. You can save different versions of your project and revert back if needed.

Remember, Overleaf has a rich set of documentation and a helpful community, so if you encounter any issues, chances are you can find the solutions in their help pages or forums. Good luck with your thesis!
