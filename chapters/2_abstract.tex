\chapter*{Abstract}
Este documento trae una plantilla de \LaTeX{} sencilla y amigable, que armé mientras hacía mi doctorado en Valencia. Con la ayuda de mis profesores Enrique Quintana Orti y Adrian Castellon, diseñé esta plantilla para ajustarse a lo que la escuela doctoral pedía en cuanto a formato para presentar la tesis.

Esta plantilla está hecha a medida siguiendo esas reglas al pie de la letra, así que todo, desde cómo se alinea el texto hasta el tamaño de la letra y los márgenes, está pensado para cumplir con lo que espera nuestra universidad. La idea es que te facilite organizar y mostrar tu trabajo de investigación de manera clara, permitiendo incluir de todo un poco: texto, imágenes, tablas y referencias.

La puedes encontrar en el repositorio que acompaña a este documento. Está pensada para ser una base sobre la que puedan trabajar los futuros doctorandos, haciendo todo el proceso de armar y entregar la tesis un poco más fácil. La he organizado de manera que todo tenga un sentido lógico, desde los primeros capítulos hasta las conclusiones y los anexos.

Te animo a que la cambies y la ajustes como mejor te parezca para tu investigación. Ya sea que necesites modificar el diseño, la estructura de los capítulos o añadir nuevos elementos, esta plantilla es flexible y se puede adaptar a diferentes áreas y temas de estudio.

Mi esperanza al compartir esta plantilla es que te sea útil y te quite un poco de encima la carga administrativa, permitiéndote concentrarte más en lo importante de tu investigación doctoral. Así que, siéntete libre de modificarla y compartirla como necesites, y muchísima suerte con tu trabajo académico.

Para más información sobre los requisitos de formato específicos, puedes consultar la página de la Universidad Politécnica de Valencia \href{https://www.upv.es/contenidos/DOCINF/info/}{aquí}.



\chapter*{Resumen}
Este documento proporciona una plantilla LaTeX estructurada y fácil de usar que fue desarrollada durante mis estudios doctorales en Valencia. Bajo la experta guía de los profesores Enrique Quintana Orti y Adrián Castellón, esta plantilla fue elaborada para cumplir con los estrictos requisitos de formato establecidos por la escuela doctoral para la presentación de tesis.

El diseño de esta plantilla se adhiere estrechamente a las directrices especificadas, asegurando que todos los elementos, desde la alineación del texto y los tamaños de las fuentes hasta los estilos de los encabezados y los márgenes, estén en conformidad con los estándares académicos esperados por la institución. La estructura está destinada a facilitar una presentación clara y organizada del trabajo académico, acomodando diversos tipos de contenido, incluyendo texto, figuras, tablas y referencias.

Disponible en el repositorio acompañante, esta plantilla pretende servir como una herramienta fundamental para futuros candidatos doctorales, agilizando el proceso de compilación y presentación de la tesis. Ha sido meticulosamente organizada para apoyar un flujo lógico de contenido, desde los capítulos introductorios hasta las observaciones finales y los apéndices.

Animo a los usuarios a adaptar y personalizar la plantilla de acuerdo a sus necesidades y preferencias de investigación individuales. Ya sea que implique ajustar el diseño, modificar la estructura de los capítulos o integrar elementos adicionales, la plantilla está diseñada con flexibilidad en mente para acomodar diversas disciplinas y enfoques de estudio.

Al compartir esta plantilla, mi esperanza es que se demuestre ser un recurso valioso, reduciendo la carga administrativa sobre los estudiantes y permitiéndoles dedicar más atención a los aspectos sustantivos de su investigación doctoral. Por favor, siéntanse libres de modificar y distribuir esta plantilla según sea necesario, y les deseo la mejor de las suertes en sus empeños académicos.
