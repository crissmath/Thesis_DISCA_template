\chapter*{Abstract}
This document brings a simple and friendly \LaTeX{} template that I put together while doing my doctorate in Valencia. With the help of my professors Enrique Quintana Orti and Adrian Castellon, I designed this template to conform to what the doctoral school required in terms of format for presenting the thesis.

This template is tailor-made, following those rules to the letter, so everything, from how the text is aligned to the size of the font and margins, is intended to meet what our university expects. The idea is that it makes it easier for you to organize and present your research work clearly, allowing for a bit of everything: text, images, tables, and references.

You can find it in the repository accompanying this document. It is meant to be a base that future doctoral candidates can work from, making the whole process of assembling and submitting the thesis a little easier. I have organized it so that everything makes logical sense, from the first chapters to the conclusions and appendices.

I encourage you to change it and adjust it as you see fit for your research. Whether you need to modify the design, the structure of the chapters, or add new elements, this template is flexible and can be adapted to different areas and study topics.

My hope in sharing this template is that it will be useful to you and take some of the administrative burden off your shoulders, allowing you to focus more on the important aspects of your doctoral research. So feel free to modify and share it as needed, and very good luck with your academic work.

For more information about specific format requirements, you can consult the Universidad Politécnica de Valencia page \href{https://www.upv.es/contenidos/DOCINF/info/}{here}. Additionally, you can find the template and additional resources in the GitHub repository \href{https://github.com/crissmath/Thesis_DISCA_template.git}{here}.





\chapter*{Resumen}
Este documento proporciona una plantilla LaTeX estructurada y fácil de usar Este documento trae una plantilla de \LaTeX{} sencilla y amigable, que armé mientras hacía mi doctorado en Valencia. Con la ayuda de mis profesores Enrique Quintana Orti y Adrian Castellon, diseñé esta plantilla para ajustarse a lo que la escuela doctoral pedía en cuanto a formato para presentar la tesis.

Esta plantilla está hecha a medida siguiendo esas reglas al pie de la letra, así que todo, desde cómo se alinea el texto hasta el tamaño de la letra y los márgenes, está pensado para cumplir con lo que espera nuestra universidad. La idea es que te facilite organizar y mostrar tu trabajo de investigación de manera clara, permitiendo incluir de todo un poco: texto, imágenes, tablas y referencias.

La puedes encontrar en el repositorio que acompaña a este documento. Está pensada para ser una base sobre la que puedan trabajar los futuros doctorandos, haciendo todo el proceso de armar y entregar la tesis un poco más fácil. La he organizado de manera que todo tenga un sentido lógico, desde los primeros capítulos hasta las conclusiones y los anexos.

Te animo a que la cambies y la ajustes como mejor te parezca para tu investigación. Ya sea que necesites modificar el diseño, la estructura de los capítulos o añadir nuevos elementos, esta plantilla es flexible y se puede adaptar a diferentes áreas y temas de estudio.

Mi esperanza al compartir esta plantilla es que te sea útil y te quite un poco de encima la carga administrativa, permitiéndote concentrarte más en lo importante de tu investigación doctoral. Así que, siéntete libre de modificarla y compartirla como necesites, y muchísima suerte con tu trabajo académico.

Para más información sobre los requisitos de formato específicos, puedes consultar la página de la Universidad Politécnica de Valencia \href{https://www.upv.es/contenidos/DOCINF/info/}{aquí}. Además, puedes encontrar la plantilla y recursos adicionales en el repositorio de GitHub \href{https://github.com/crissmath/Thesis_DISCA_template.git}{aquí}.
